%!TEX TS-program = xelatex
%!TEX encoding = UTF-8 Unicode
% Awesome CV LaTeX Template for Cover Letter
%
% This template has been downloaded from:
% https://github.com/posquit0/Awesome-CV
%
% Authors:
% Claud D. Park <posquit0.bj@gmail.com>
% Lars Richter <mail@ayeks.de>
%
% Template license:
% CC BY-SA 4.0 (https://creativecommons.org/licenses/by-sa/4.0/)
%


%-------------------------------------------------------------------------------
% CONFIGURATIONS
%-------------------------------------------------------------------------------
% A4 paper size by default, use 'letterpaper' for US letter
\documentclass[11pt, a4paper]{awesome-cv}

% Configure page margins with geometry
\geometry{left=1.4cm, top=.8cm, right=1.4cm, bottom=1.8cm, footskip=.5cm}

% Specify the location of the included fonts
\fontdir[fonts/]

% Color for highlights
% Awesome Colors: awesome-emerald, awesome-skyblue, awesome-red, awesome-pink, awesome-orange
%                 awesome-nephritis, awesome-concrete, awesome-darknight
\colorlet{awesome}{awesome-red}
% Uncomment if you would like to specify your own color
% \definecolor{awesome}{HTML}{CA63A8}

% Colors for text
% Uncomment if you would like to specify your own color
% \definecolor{darktext}{HTML}{414141}
% \definecolor{text}{HTML}{333333}
% \definecolor{graytext}{HTML}{5D5D5D}
% \definecolor{lighttext}{HTML}{999999}

% Set false if you don't want to highlight section with awesome color
\setbool{acvSectionColorHighlight}{true}

% If you would like to change the social information separator from a pipe (|) to something else
\renewcommand{\acvHeaderSocialSep}{\quad\textbar\quad}


%-------------------------------------------------------------------------------
%	PERSONAL INFORMATION
%	Comment any of the lines below if they are not required
%-------------------------------------------------------------------------------
% Available options: circle|rectangle,edge/noedge,left/right
%\photo[circle,noedge,left]{./examples/profile}
\name{Rinat}{Tagirov}
\position{Research Associate}
\address{Imperial College London, Blackett Laboratory (Astrophysics Group), Prince Consort Road, London SW7 2AZ, UK}

\mobile{+44 (0)74-3640-1732}
\email{tagirovrinat@gmail.com}
%\homepage{www.posquit0.com}
%\github{posquit0}
\github{rtagirov}
\linkedin{rinat-tagirov-7628b790}
% \gitlab{gitlab-id}
% \stackoverflow{SO-id}{SO-name}
% \twitter{@twit}
\skype{tagirovrinat}
% \reddit{reddit-id}
% \extrainfo{extra informations}

%\quote{``Be the change that you want to see in the world."}

%-------------------------------------------------------------------------------
% LETTER INFORMATION
% All of the below lines must be filled out
%-------------------------------------------------------------------------------
% The company being applied to
\recipient
{Dr. Sarah Martell}
{University of New South Wales\\
 School of Physics\\
 Department of Astrophysics\\
 Kensington NSW 2033, Australia}
% The date on the letter, default is the date of compilation
\letterdate{\today}
% The title of the letter
%\lettertitle{Job Application for Postdoc Research Fellow in Galactic Archaeology}
% How the letter is opened
\letteropening{Dear Dr. Martell,}
% How the letter is closed
\letterclosing{Sincerely,}
% Any enclosures with the letter
%\letterenclosure[Attached]{Curriculum Vitae}


%-------------------------------------------------------------------------------
\begin{document}

% Print the header with above personal informations
% Give optional argument to change alignment(C: center, L: left, R: right)
%\makecvheader[R]

% Print the footer with 3 arguments(<left>, <center>, <right>)
% Leave any of these blank if they are not needed
\makecvfooter
%  {\today}
  {}
  {Rinat Tagirov~~~·~~~Cover Letter}
  {}

% Print the title with above letter informations
\makelettertitle

%-------------------------------------------------------------------------------
%	LETTER CONTENT
%-------------------------------------------------------------------------------
\begin{cvletter}

Please accept my application for the Postdoc Research Fellow position in galactic archaeology at the UNSW.

I am currently a research associate with the Astrophysics Group at Imperial College London
working in the field of NLTE radiative transfer.
My work is concerned with the NLTE Spectral SYnthesis code (NESSY) development and its application to
solar and stellar brightness variability modeling.
It started back in 2011 when I graduated from the Astronomical Department of Saint Petersburg State University
and commenced my PhD studies at Physical-Meteorological Observatory Davos,
where the code had been used for solar irradiance calculations.
The purpose of my project was to apply the code for understanding the connection between the UV and radio variabilities of the solar spectrum.
In order to achieve this goal, the code had to be upgraded and tested.
Hence, my PhD dissertation was naturally divided into three parts: code development, its verification and application.

The first part was devoted to upgrading the code's NLTE block.
NESSY was originally created for synthesizing the spectra emergent from Wolf-Rayet stars and its NLTE scheme
was inefficient for the solar conditions, especially in the upper parts of the solar atmosphere
where radio and UV radiation forms. After the upgrade the code became much faster, more reliable 
and applicable to both solar and stellar spectrum calculations.
In the second part, I tested the code by comparing the center-to-limb variations of solar brightness calculated with it
to the ones obtained from the solar eclipse observations by PREMOS instrument on-board the PICARD satellite.
To this end, a procedure for extracting the center-to-limb variations from the eclipse observations was developed.
Finally, in the third part, I applied NESSY together with HMI/SDO observations to model the solar variability in the UV and radio
spectral domains and analyzed the correlation between them.

I published the results of the first part of my thesis in Astronomy \& Astrophysics 
and defended my dissertation at ETH Z{\"u}rich in the beginning of 2017.
At that time I was already working at Imperial College, continuing to develop
the code in collaboration with Dr. Yvonne Unruh to make it suitable
for the so-called 1.5D solar and stellar variability calculations. 
In these calculations the advantages of 1D-NLTE radiative transfer are
combined with the ability of the current 3D-MHD models to capture the dynamics of stellar interiors.
I have made NESSY capable of switching between NTLE and LTE radiative transfer regimes for different chemical elements,
and now I am working on merging the NLTE block of NESSY with the ATLAS9 spectral synthesis code.
The objective of these projects is to make NESSY fast enough to handle the computationally taxing 1.5D calculations.
My work has been accompanied by teaching, which gives me an opportunity to explore other areas of physics,
and by frequent collaboration visits to Max-Planck Institute for Solar System Research,
where I contribute to the efforts of the ERC research group SOLVe (Connecting Solar and Stellar Variabilities)
led by my former supervisor Dr. Alexander Shapiro.

My project will be brought to completion by the end of the upcoming summer.
I have been working in my current field for quite a long time and I am currently looking 
for an opportunity to apply myself in another area of astrophysics.
As a specialist in NLTE stellar radiative transfer calculations I have some experience in using spectroscopic data sets, albeit not very large.
That being said, I have acquired solid programming skills and I am always open to learn new ones, which makes me confident that
I will be successful in expanding my data-handling abilities.

I look forward to discussing the Postdoc Research Fellow position in galactic archaeology with you at an interview.

Sincerely,\\
Rinat Tagirov

\end{cvletter}


%-------------------------------------------------------------------------------
% Print the signature and enclosures with above letter informations
%\makeletterclosing

\end{document}
