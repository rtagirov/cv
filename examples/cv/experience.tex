%-------------------------------------------------------------------------------
%	SECTION TITLE
%-------------------------------------------------------------------------------
\cvsection{Experience}


%-------------------------------------------------------------------------------
%	CONTENT
%-------------------------------------------------------------------------------
\begin{cventries}

%---------------------------------------------------------
  \cventry
    {Research Associate} % Job title
    {Imperial College London} % Organization
    {London, UK} % Location
    {Oct. 2016 --- PRESENT} % Date(s)
    {
      \begin{cvitems} % Description(s) of tasks/responsibilities
        \item {Radiative transfer code development, solar spectrum modeling, solar irradiance variability modeling}
%        \item {Developed NESSY for its implementation in 1.5D solar irradiance calculations, which included:}
%            \begin{itemize}
%                \item {implementation of mixed NLTE/LTE calculations in NESSY;}
%                \item {merging the ATLAS9 code with the NLTE block of NESSY.}
%            \end{itemize}
      \end{cvitems}
    }

%---------------------------------------------------------
  \cventry
    {PHD Student} % Job title
    {Physical-Meteorological Observatory Davos} % Organization
    {Davos, Switzerland} % Location
    {Sep. 2011 --- Sep. 2016} % Date(s)
    {
      \begin{cvitems} % Description(s) of tasks/responsibilities
        \item {Radiative transfer code development, solar spectrum modeling, solar irradiance variability modeling}
%        \item {Implemented accelerated $\Lambda$-iterations in the stellar radiative transfer code NESSY.}
%        \item {Improved a method for derivation of CLVs of solar brightness from solar eclipse observations.}
%        \item {Applied this method to PREMOS/PICARDS solar eclipse data.}
%        \item {Compared the derived CLVs to the ones calculated with NESSY in order to test 1D models of solar atmosphere.}
%        \item {Applied NESSY to calculate and analyze the facular and spot contrasts.}
%        \item {Used these contrasts to model the solar irradiace in UV and radio and analyze the correlation between the two.}
      \end{cvitems}
    }

%---------------------------------------------------------
  \cventry
    {Research Assistant} % Job title
    {Ioffe Physical-Technical Institute} % Organization
    {Saint-Petersburg, Russia} % Location
    {Sep. 2010 - Jun. 2011} % Date(s)
    {
      \begin{cvitems} % Description(s) of tasks/responsibilities
        \item {Physics of interstellar medium in the early Universe}
%        \item {Improved a method for calculating particle concentration in molecular clouds at high redshifts.}
%        \item {Using this method together with observations of carbon atom fine-structure lines:}
%            \begin{itemize}
%                \item {calculated the CMB temperature in two molecular clouds associated with quasars J0812+3208 and Q1232+082;}
%                \item {calculated the hydrogen molecular fraction in these clouds.}
%            \end{itemize}
%        \item {Estimated the UV radiation background and electron concentration in their inner parts.}
      \end{cvitems}
    }

%---------------------------------------------------------
\end{cventries}
